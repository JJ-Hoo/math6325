\documentclass{article}

%packages
\usepackage[utf8]{inputenc}
\usepackage{amsmath}
\usepackage{amssymb}
\usepackage{tikz-cd}
\usepackage{amsfonts}
\usepackage{float}
\usepackage{amsthm}
\usepackage{outlines}
\usepackage{graphicx}
\usepackage{fullpage}
\usepackage{color}
\usepackage{hyperref}
\usepackage{mathtools}
\usepackage{easyReview}
\hypersetup{
	colorlinks=true,
	linkcolor=blue,
	urlcolor=red,
	linktoc=all
}
\definecolor{dg}{rgb}{0.0, 0.5, 0.0}
\setlength\parindent{0pt}


%Commands
\newcommand{\R}{\mathbb{R}}
\newcommand{\C}{\mathbb{C}}
\newcommand{\N}{\mathbb{N}}
\newcommand{\Q}{\mathbb{Q}}
\newcommand{\Z}{\mathbb{Z}}
\newcommand{\D}{\mathbb{D}}
\newcommand{\HH}{\mathbb{H}}

%Theorems, etc. (shared counters)
\newtheorem{theorem}[subsection]{Theorem}
\newtheorem{example}[subsection]{Example}
\newtheorem{proposition}[subsection]{Proposition}
\newtheorem{lemma}[subsection]{Lemma}
\newtheorem{definition}[subsection]{Definition}

%Math Operators
\DeclareMathOperator{\calg}{\textsf{CAlg}^{*}_{\mathbb{C}}}
\DeclareMathOperator{\linf}{L^\infty}
\DeclareMathOperator{\preems}{\textsf{PreEMS}}
\DeclareMathOperator{\strictems}{\textsf{StrictEMS}}
\DeclareMathOperator{\ban}{\textsf{Ban}}
\DeclareMathOperator{\banop}{\textsf{Ban}^{op}}
\DeclareMathOperator{\ball}{\textsf{Ball}}

\title{Category Theory Problem Sets and Solutions}
\author{Fall 2022}
\begin{document}
	
	\maketitle
	\tableofcontents
	\newpage
	\section{Problem Set 1}
	\subsection*{Assignments}
	\begin{itemize}
		\item Problem 1 - Unclaimed
		\item Problem 2 - Emilio Verdooren
		\item Problem 3 - Emilio Verdooren
		\item Problem 4 - Orin Gotchey
		\item Problem 5 - Unclaimed
		\item Problem 6 - Alan Bohnert
		\item Problem 7 - James
		\item Problem 8 - James
		\item Problem 9 - Unclaimed
		\item Problem 10 - Unclaimed
		\item Problem 11 - Unclaimed
	\end{itemize}
	
	\subsection{Problem 4 - Orin Gotchey}
	\subsubsection{Measurable Spaces as a Category}
	\begin{definition}{$\sigma{}$-algebras}
		Let $X$ be a set.  Let $\Omega$ be any subset of $\mathcal{P}(X)$ satisfying the following conditions:
		\begin{itemize}
			\item $X\in\Omega_{X}$
			\item For each $E\in\Omega$, $X\backslash E\in\Omega_{X}$
			\item For any index $I: \mathbb{N}\rightarrow\Omega_{X}$, $(\cup_{n\in\mathbb{N}}I(n))\in\Omega_{X}$
		\end{itemize} $\Omega$ is called a $\sigma$\textit{-algebra} on $X$, the pair $(X,\Omega_{X})$ a \textit{measurable space}, the elements of $\Omega_{X}$ the \textit{measurable subsets} of $X$.  
	\end{definition}
	It follows immediately that $\varnothing\in\Omega_{X}$, and that $\Omega_{X}$ is closed under countable intersection.\\
	\begin{definition}{Measurable Maps}The maps $f: X\rightarrow Y$ between measurable spaces which have the following property: 
		\begin{equation*}
			\forall E\subset X: (f(E)\in\Sigma\implies E\in\Omega) 
		\end{equation*}
		are called \textit{measurable maps} or \textit{measurable functions}.
	\end{definition}
	Let $\texttt{Meas}$ be the category specified as follows:
	\begin{outline}
		\1 Objects are the measurable spaces $(X,\Omega)$
		\1 Morphisms are the measurable functions.
	\end{outline}
	Then, given any morphism $f: X\rightarrow{}Y$,
	\begin{equation*}
		f\circ\texttt{Id}_X = f
	\end{equation*}
	\begin{equation*}
		\texttt{Id}_Y\circ{}f = f
	\end{equation*}
	Associativity follows from the fact that the composition of functions on the underlying sets is associative.\\Given two composable morphisms, say $f: X\rightarrow{}Y$ and $g:Y\rightarrow{}Z$, consider the composition $g\circ f: X\rightarrow{}Z$, and let $\gamma\in\Omega_{Z}$.  Then:
	\begin{equation*}
		g^{-1}(\gamma) \in \Omega_{Y}
	\end{equation*}
	\begin{equation*}
		f^{-1}(g^{-1}(\gamma)) = (g\circ f)^{-1}(\gamma) \in\Omega_{X}
	\end{equation*}.
	Thus, we have that \texttt{Meas} is a category.
	
	\subsubsection{Enhanced Measurable Spaces}
	Let $(X,\Omega_X)$ be a topological space.  Then $\mathcal{P}(X)$ forms a Boolean commutative ring with the operations $\cap$ and $\triangle$ as multiplication and addition, respectively, and of which $\Omega_X$ is a subring.
	Define an \textit{enhanced measurable space} as a triple $(X,\Omega_X,N_X)$, where $(X,\Omega_X)$ form a measurable space, and $N_X$ is a $\sigma$-ideal of $\Omega_{X}$ (recall: a $\sigma$-\textit{ideal} is an ideal which is closed under \textit{countable} addition).  A \textit{negligible set} in $X$ is some subset of $N_{X}$.\\The \textit{measurable maps} $f: (X,\Omega_X, N_X)\rightarrow(Y,\Omega_Y,N_Y)$ are maps of sets: $f: X_f\rightarrow{}Y$, where $X_f\subset{}X$, such that $f$ obeys the following conditions (which are verified for the identity maps in the subpoints where $X=Y$):
	\begin{enumerate}
		\item The set $X\backslash{}X_f$ is negligible
		\begin{itemize}\item $X=X_{\texttt{Id}_X}$ and $X\backslash{}X_{\texttt{Id}_X}=\varnothing\in{}N_X$, by definition of ideal. \end{itemize}
		\item For any $m_{y}\in\Omega_Y$, there exists a set $m_{x}$ such that $f^{-1}(m_y)\triangle{}m_x$ is negligible
		\begin{itemize}\item Given $m_{x}$, $\texttt{Id}^{-1}_X(m_x)\triangle{}m_X = m_x\triangle{}m_x = \varnothing\in{}N_X$ \end{itemize}
		\item For any $n_y\in N_Y$, the set $f^{-1}(n_y)$ is negligible.
		\begin{itemize} \item $\texttt{Id}_X(n_x) = n_x$ \end{itemize}
	\end{enumerate}
	We cannot define composition of morphisms strictly as composition of underlying maps, because there is no guarantee, e.g., for two maps between enhanced measurable spaces $f:X\rightarrow{}Y,\,g:Y\rightarrow{}Z$, that $\textit{Im}f\subset{}Y_g$.  Thus, we restrict the domain of the composition to:
	\begin{equation*}
		X_{g\circ{}f} := f^{-1}(Y_g)
	\end{equation*}.
	However, it is clear by inspection that composition of morphisms retains associativity.
	Then,
	\begin{equation*}
		X\backslash{}X_{g\circ{}f} = X\backslash{}f^{-1}(Y_g) = (f^{-1}(Y\backslash(Y_g)))
	\end{equation*}
	The negligibility of the above quantity then follows from the definition of $f$.\\
	Furthermore, given $m_z\in\Omega_Z$, we have that $(g\circ{}f)^{-1}(m_z) = f^{-1}(g^{-1}(m_z))$.  Since $g$ is measurable (\textit{why?}) and since $f$ is presumed to satisfy (2), $g\circ{}f$ satisfies (2).\\(3) is clearly transitive.\\Thus, enhanced measurable spaces and measurable maps form a category.
	\subsubsection{Equality Almost Everywhere}
	Two parallel morphisms $f,g: (X,\Omega_X, N_X)\rightarrow{}(Y,\Omega_Y,N_Y)$ are "equal almost everywhere" if the set $\{x\in X_f\cap{}X_G : f(x)\neq{}g(x)\}$ is negligible.  Let "$f$ and $g$ are equal almost everywhere" be denoted $f\sim{}g$.  Claim: $\sim$ defines an equivalence relation.
	\begin{itemize}
		\item Reflexivity: A function differs from itself on the empty set ($\varnothing$), which is negligible (see above)
		\item Symmetry: Note that the symbols $f$ and $g$ in the definition of equality almost everywhere are symmetric
		\item Transitivity: If $f\sim{}g$ and $g\sim{}h$ for parallel morphisms $f,g,$ and $h$, then 
		\begin{equation*}
			\{x\in X_f\cap{}X_h:f(x)\neq h(x)\}\subset{}(\{x\in X_f\cap{}X_g: f(x)\neq g(x)\}\cup{}\{x\in{}X_g\cap{}X_h: g(x)\neq{}h(x)\})
		\end{equation*}, and $N$ is closed under countable unions and taking subsets, so the left hand side of the above is negligible.
	\end{itemize}
	Furthermore, this equivalence relation is compatible with composition.  Assume that there are morphisms $f,f': X\rightarrow{}Y$ and $g,g': Y\rightarrow{}Z$. such that $f\sim{}f'$ and $g\sim{}g'$.  We're interested in the set
	\begin{multline*}
		\{ x\in{}X_{g\circ{}f}\cap X_{g'\circ{}f'} : (g\circ{}f)(x)\neq(g'\circ{}f')(x) \} \subset{} \{x\in{}X_f\cap{}X_{f'} : f(x)\neq{}f'(x)\}\\ \cup{}f^{-1}(\{y\in{}Y_g\cap{}Y_g': g(y)\neq{}g'(y)\})
	\end{multline*}
	This set is the union of two negligible sets.
	\subsubsection{Hom-sets mod an Equivalence Relation}
	Suppose that for every pair of objects $X,Y$ in a category $C$, we are given (e.g. by the above) an equivalence relation $R_{X,Y}$ on $C(X,Y)$ that is compatible with composition (i.e. if $f\sim_Rf'$ and $g\sim_Rg'$ then $(g\circ{}f)\sim_R(g'\circ{}f')$.  We identify all morphisms in $C$ between any two objects $X$ and $Y$ which relate through $R_{X,Y}$.  Composition of equivalence classes of $\sim{}$ does not depend on choice of representative: this is exactly compatibility with $\circ{}$\\Verifying that the proper morphisms are unital and associative are gifted as simple exercises to the reader ;)
	
	\subsection{Problem 5 - Bradley Vigil}
Fix a category $\textbf{C}$. A $\textit{section}$ of a morphism $f:X\to Y$ in $\textbf{C}$ is a morphism $g:Y\to X$ such that $fg=\text{id}_{Y}$. Give an example of a category $\textbf{C}$ such that all epimorphisms have sections. Give an example of a category $\textbf{C}$ and an epimorphism $f$ in $\textbf{C}$ that does not have a section.

\begin{lemma}
In $\mathbf{Mod}$-$R$ (or $R$-$\mathbf{Mod}$), the category of right $R$-modules (left $R$-modules) over a ring $R$, a morphism $f:M\to N$ in $\mathbf{Mod}$-$R$ (or $R$-$\mathbf{Mod}$) is an epimorphism if and only if $f$ is surjective.
\end{lemma}

\begin{proof}
Let $M$, $N$, and $L$ be right $R$-modules. We first prove the backwards direction. That is, we suppose that $f:M\to N$ is a surjection and show that $f$ is an epimorphism. Let $g_{1},g_{2}:N\to L$ be morphisms, if $f:M\to N$ is a surjection and $g_{1}\circ f=g_{2}\circ f$ it follows that $g_{1}=g_{2}$. To see this, we note that from the surjectivity of $f$ we have $\forall n\in N$, $\exists m\in M$ such that $f(m)=n$. Therefore $\forall n\in N$, $g_{1}(n)=g_{1}(f(m))=g_{2}(f(m))=g_{2}(n)$. Thus, $f$ is an epimorphism.\\
Conversely, suppose that $f:M\to N$ is an epimorphism. Note that since we are in $\mathbf{Mod}$-$R$, im($f$) is a submodule of $N$ hence $N$/im($f$) is a module and the canonical morphism $g_{1}:N\to N/\text{im}(f)$ which maps $n$ $\mapsto$ $n$+im($f$) is well-defined. Take $g_{2}:N\to N/\text{im}(f)$ to be the zero morphism, that is $g_{2}(n)=0+\text{im}(f)$ $\forall n\in N$. Since $f$ is an epimorphism we have that $g_{1}=g_{2}$, that is $g_{1}(n)=g_{2}(n)=0+\text{im}(f)$ $\forall n\in N$. Therefore, it follows that $n+\text{im}(f)=0+\text{im}(f) \in N/\text{im}(f)$ $\forall n\in N$. That is to say $[n]=[0] \Longrightarrow n=0 \in N/\text{im}(f)$, i.e. the representatives in the quotient module must be equal. Thus, we see that $n$ is zero in the quotient module $N/\text{im}(f)$ which implies that $n\in \text{im}(f)$ $\forall n\in N$. Hence, $f$ is a surjection. The proof for $R$-$\mathbf{Mod}$ follows identically.
\end{proof}

\begin{proposition}
In the category of abelian groups, $\mathbf{Ab}$, not all epimorphisms have sections.
\end{proposition}

\begin{proof}
Consider $\mathbf{Ab}$, the category of abelian groups, and let $f:\mathbb{Z}\to \mathbb{Z}/2\mathbb{Z}$ which maps $z \mapsto [z]_{2}$. Then $f$ is a morphism in $\mathbf{Ab}$ since $f$ is a group homomorphism. Recall that abelian groups are $\mathbb{Z}$-modules and since $f$ is surjective, it follows from the above lemma that $f$ is an epimorphism in $\mathbf{Ab}$. Note that the only morphism $g:\mathbb{Z}/2\mathbb{Z}\to \mathbb{Z}$ is the zero morphism because $\mathbb{Z}/2\mathbb{Z}$ has the property that $1+1=0$. Therefore, $f\circ g=0$. Hence, $f$ does not have a section in $\mathbf{Ab}$.
\end{proof}

\begin{proposition}
In the category of $K$-vector spaces, $\mathbf{Vect}_{K}$, all epimorphisms have a section.
\end{proposition}

\begin{proof}
Moreover, consider $\mathbf{Vect}_{K}$, the category of $K$-vector spaces. Note that $K$-vector spaces are left $K$-modules over the ring $K$. Therefore, by the above lemma, any epimorphism $f$ in $\mathbf{Vect}_{K}$ is a surjection. In particular, let $V$ and $W$ be $K$-vector spaces, and let $f:V\to W$ be an epimorphism in $\mathbf{Vect}_{K}$ then $f$ is a surjection. Note that since $W$ is a $K$-vector space, and in particular a free module, it is generated by it's basis elements. Let $\mathcal{W}$ be a basis for $W$, from the surjectivity of $f$ we have that $\forall w_{i}\in \mathcal{W}$, $\exists v_{i}\in V$ such that $f(v_{i})=w_{i}$ $\forall i$ in an indexing set $\textit{I}$. Now define $g:W\to V$ by $g(w_{i})=v_{i}$. Extending by K-linearity it follows that $\forall k_{i}\in K$

\[g(\sum\limits_{i\in \textit{I}} k_{i}w_{i})=\sum\limits_{i\in \textit{I}} k_{i}v_{i}=\sum\limits_{i\in \textit{I}} k_{i}g(w_{i})\]
That is, $g$ is $K$-linear. Therefore, $g$ is a morphism in $\mathbf{Vect}_{K}$ and since every element $w\in W$ can be written as a linear combination of $w_{i}\in \mathcal{W}$ and both $g$ and $f$ are morphisms in $\mathbf{Vect}_{K}$, we have that $(f\circ g)(w)=f(g(w))=w$ $\forall w\in W$. From the arbitrariness of $f$ it follows that every epimorphism in $\mathbf{Vect}_{K}$ has a section. 
\end{proof}
	
	\subsection{Problem 6 - Alan Bohnert}
	\subsubsection{Question}
	Fix a category \textbf{C}. 
	A \textit{bimorphism} in \textbf{C} is a morphism $f$ that is simultaneously a monomorphsim and an epimorphism. Is any isomorphism a bimorphism? 
	Give and example of a category \textbf{C} and a bimorphism $f$ in \textbf{C} that is not an isomorphism.
	
	\subsubsection{Solution}
	In any category \textbf{C} every isomorphism is a bimorphism.
	
	\begin{proof}
		Let $f:X\xrightarrow{}Y$ be an isomorphism in \textbf{C}. 
		Then there exists a morphism $g:Y\xrightarrow{}X$ in \textbf{C} such that 
		\begin{center}
			$gf=$Id$_X$ and $fg=$Id$_Y$.
		\end{center}
		
		To show $f$ is a monomorphism let $h,k:W \rightrightarrows X$ and $fh=fk$.
		It follows that $gfh=gfk$ for the $g$ given above.
		Therefore Id$_X h=$Id$_X k$ and so $h=k$ tells us $f$ is a monomorphism.
		
		To show $f$ is an epimorphism let $m,n:Y \rightrightarrows Z$ and $mf=nf$.
		Composing with the $g$ we know $mfg=nfg$.
		Consequently $m$Id$_Y=n$Id$_Y$ and $m=n$ tells us $f$ is an epimorphism.
		Therefore $f$ is a bimorphism.
	\end{proof}
	
	\vspace{2mm}
	
	\noindent Let \textbf{C} be the category \textbf{Ring} and let $f:\Z \xhookrightarrow[]{} \Q$ the inclusion map.
	We claim $f$ is a bimorphism but not an isomorphism.
	
	\begin{proof}
		To show $f$ is a monomorphism let $h,k:W \rightrightarrows \Z$ and $f\circ h(w)=f \circ k (w)$ $\forall w\in W$.
		Since $f$ is injective
		\begin{center}
			$h(w)=f\circ h(w)=f\circ k(w)=k(w)$.
		\end{center}
		Therefore $h(w)=k(w)$ and $f$ is a monomorphism.
		
		To show $f$ is an epimorphism let $m,n:\Q \rightrightarrows S$ such that $m\circ f(x)=n\circ f(x)$ $\forall x\in \Q$.
		Since $f$ is injective, we know $m(z)=n(z)$ $\forall z\in\Z$.
		Seeking a contraction, suppose there exists $\frac{a}{b}\in \Q$ such that $m(\frac{a}{b})\neq n(\frac{a}{b})$.
		Given $m$ and $n$ are ring homomorphisms we know
		\begin{center}
			$m(a)m(b^{-1})=m(\frac{a}{b}) \neq n(\frac{a}{b})=n(a)n(b^{-1})$.
		\end{center}
		Given $b$ is an invertible integer and $m(b)=n(b)$ we can multiply on the right and retain the inequality.
		Thus, 
		\begin{center}
			$m(a)m(b^{-1})m(b)\neq n(a)n(b^{-1})n(b)$
		\end{center} 
		and as ring homomorphisms we have
		\begin{center}
			$m(a)=m(a)m(b^{-1}b)\neq n(a)n(b^{-1}b)=n(a)$.
		\end{center}
		Therefore $f$ is a epimorphism.
		
		To show $f$ is not an isomorphism we note $\frac{1}{3}\in \Q$ has no preimage in $\Z$.
	\end{proof}
\section{Problem Set 2}
\subsection*{Assignments}
\begin{itemize}
	\item Problem 1 - Orin Gotchey
	\item Problem 2 - James
	\item Problem 3 - Bradley
	\item Problem 4 - Alan
	\item Problem 5 - Mason
	\item Problem 6 - Emilio
\end{itemize}
\subsection{Problem 1 - Orin Gotchey}
\begin{lemma}{Existence and Uniqueness of Borel $\sigma$-Algebras.}
	Let $X$ be a topological space.  Then there exists a unique $\sigma$-algebra, $\Omega$ on $X$ which contains all open subsets of $X$ and which is the smallest among such $\sigma$-algebras with respect to inclusion. 
\end{lemma}
\begin{proof}
	Let $\Sigma$ be the collection of all $\sigma$-algebras on $X$ which contain all open subsets of $X$.  $\Sigma$ contains $\mathcal{P}(X)$, and thus is nonempty.	Let
	\begin{equation*}
	\Omega := \cap_{x\in\Sigma}x
	\end{equation*}
	Clearly, $X\in\Omega$. Given an index $I: \mathbb{N}\rightarrow\Omega$, such that for every natural $n$, $I(n)\in\Omega$, we have that $I(n)\in{x},\;\forall{x}\in\Sigma$, whence it follows that $\cap_{n\in\mathbb{N}}I(n)\in{x},\;\forall{x}\in\Sigma$.  Therefore, $\cap_{n\in\mathbb{N}}I(n)\in{\Omega}$.  By a similar argument, for any $E\in\Omega$, $X\backslash{E}\in\Omega$.  Thus, $\Omega$ contains all open subsets of $X$, and is indeed inferior to any other $\sigma$-algebra with this property.  
\end{proof}
\begin{definition}
	A \textit{complex *-algebra} $A$ is a complex algebra, equipped with a complex-antilinear operation $*:A\rightarrow{A}$ obeying the following:
	\begin{equation*}
		(ab)^* = b^*a^*
	\end{equation*}
	\begin{equation*}
		1^* = 1
	\end{equation*}
	\begin{equation*}
		(a^*)^* = a
	\end{equation*}
\end{definition}
\begin{definition}
	A complex-valued morphism $f: X\rightarrow{}\C$ (on some topological space $X$) is called "bounded" if it factors through some bounded subset of $\C$.  That is, there exists some subset $C\in\C$ which is contained in some open ball, and some map $\bar{f}$ which makes the following diagram commute:
\end{definition}
\begin{center}
\begin{tikzcd}
	X \arrow[r,"f"] \arrow[d,"\bar{f}",dashed,swap] & \C \\
	C \arrow[ur,"\iota",swap]
\end{tikzcd}
\end{center}
\begin{lemma}
	Given an enhance measurable set $(X,\Omega_X,N_X)$, the set of all bounded morphisms $\{f: (X,\Omega_X,N_X)\rightarrow{}(\C,\Omega_C,\{\varnothing\})\}$ is a complex *-algebra.
\end{lemma}
\begin{proof}
	The zero morphism $0_X$ acts as the additive identity.  Addition, multiplication, and involution are pointwise.  Everything else follows by inspection.
\end{proof}
\begin{proposition}
	Together with complex algebra homomorphisms: $f: A\rightarrow{B}$ satisfying $f(a^*) = f(a)^*$, and objects: commutative complex *-algebras, $\calg$ is a category.
\end{proposition}
\begin{proof}
	Let $\forall a,b,c\in\textsf{Obj}(\calg),\;f\in\calg(a,b),\;g\in\calg(b,c)$ then:
	\begin{itemize}
		\item $\exists \texttt{id}_{a}:a\rightarrow{a}$ given by $\texttt{id}_a(x)=x$ satisfies $\texttt{id}_a(x^*) = x^* = \texttt{id}_a(x)^*$, and which is clearly a $\C$-algebra homomorphism
		\item $g\circ{f}$ satisfies $(g\circ{f})(x)^* = g(f(x))^* = g(f(x)^*) = g(f(x^*)) = g\circ{f}(x^*)$, and is clearly a $\C$-algebra homomorphism.
		\item The composition of underlying sets is associative.
	\end{itemize}
\end{proof}
	Let $\textsf{L}^\infty: \textsf{PreEMS}^{op}\rightarrow{\calg}$ send an enhanced measurable space to the complex *-algebra of bounded morphisms: $(X,\Omega_X,N_X) \mapsto (\linf(X): \{\phi: (X,\Omega_X,N_X)\rightarrow(\C,\Omega_\C, \{\varnothing\})|\,\phi\,\text{bounded}\})$, and which sends an enhanced measurable morphism $f: (X,\Omega_X,N_X)\rightarrow{}(Y,\Omega_Y,N_Y)$ to \begin{equation*}
\textsf{L}^\infty(f) : (\linf(Y) : \{\phi:(Y,\Omega_Y,N_Y)\rightarrow(\C,\Omega_{\C},N_\C)\})\rightarrow{}(\linf(X) : \{\psi : (X,\Omega_X,N_X)\rightarrow(\C,\Omega_\C,N_\C)\})
	\end{equation*}
given by:
\begin{equation*}
	(\linf(f))(\phi) = (\phi\circ{}f)
\end{equation*}
\begin{proposition}
	$\linf$ is a contravariant functor
\end{proposition}
\begin{proof} We need to show the following:
\begin{enumerate}
	\item $\linf(f)$ defines a morphism in $\calg$ i.e. a complex algebra homomorphism which respects involution.
	\item $\linf$ respects identity
	\item $\linf$ respects composition
\end{enumerate}
For (1), given an $f: X\rightarrow{Y}$, $\phi,\;\psi\in\linf(Y)$, and $c\in\C$
\begin{equation}
\begin{split}
	&\linf(f) : \linf(Y)\rightarrow{}\linf(X)\\
	&\linf(f)(0_Y) = 0_X\\
	&\linf(f)(\phi+\psi) = (\phi+\psi)\circ(f) = (\phi\circ{f}) + (\psi\circ{f}) = \linf(f) (\phi) + \linf(f)(\psi)\\
	&\linf(f)(\phi\cdot\psi) = (\phi\cdot\psi)\circ{f} = (\phi\circ{f})\cdot(\psi\circ{f}) = \linf(f)(\phi)\cdot\linf(f)(\psi)\\
	&c\cdot\linf(f)(\phi) = c\cdot(\phi\circ{f}) = (c\cdot\phi)\circ{f} = \linf(f)(c\cdot\phi)\\
	&\linf(f)(\phi^*) = (\phi^*)\circ{f} = (\phi\circ{f})^* = \linf(f)(\phi)^*
\end{split}
\end{equation}
For (2),
\begin{equation}
	\linf(\texttt{id}_X)(\phi) = (\phi\circ\texttt{id}_X) = \phi \implies \linf(\texttt{id}_X) = \texttt{id}_{\linf(X)} 
\end{equation}
For (3), we give two morphisms $f: X\rightarrow Y\;,g:Y\rightarrow Z$ in $\preems$.  Then for any $\phi\in\linf(Z)$
\begin{equation}
	\linf(g\circ{}f)(\phi) = \phi\circ(g\circ{f}) = (\phi\circ{g})\circ{f} = \linf(f)(\phi\circ{g}) = \linf(f)(\linf(g)(\phi)) = (\linf(f)\circ{}\linf(g))(\phi)
\end{equation}
\end{proof}
\begin{lemma}
	Let $\textsf{C}$ be a category with an equivalence relation $R$ on its set of morphisms, and let $F$ be some functor from $\textsf{C}/R$ to another category $\textsf{D}$.  Then precomposing with the functor $\textsf{C}\xrightarrow{\Pi}\textsf{C}/R$ gives a bijection between functors $\textsf{C}/R\xrightarrow{F}\textsf{D}$ and functors $\textsf{C}\xrightarrow{G}\textsf{D}$ with the property that $\alpha\sim_R\beta\Rightarrow{}G(\alpha) = G(\beta)$
\end{lemma}
\begin{proof}
	If $\alpha\sim_R\beta$, then $\Pi(\alpha)=\Pi(\beta)$, so $(F\circ\Pi)(\alpha) = (F\circ\Pi)(\beta)$.  For the other direction, let $G:\textsf{C}\rightarrow{\textsf{D}}$ and $R$ be given as above.  Then, let $F: \textsf{C}/R\rightarrow{D}$ be identical to $G$ on objects, and let the image of an equivalence class of morhpisms in $\textsf{C}/R$ be the image of any of its representatives under $G$.  This is well-defined, because any two morphisms in the same equivalence class of $R$ will have the same image under $G$.
\end{proof}
\begin{definition}
	Let R be an equivalence relation defined on the $\textsf{Hom}$-sets of $\preems$.  Define the category $\strictems$, whose objects are the same as $\preems$, and whose morphisms are equivalence classes of morphisms under $R$ sharing domain and codomain.
\end{definition}
It follows from the foregoing observations that $\strictems$ is, in fact, a category.  Extend the $\linf$ to a functor from $\strictems$ to $\calg$ as follows:
\begin{itemize}
	\item Send an enhanced measurable space $(X,\Omega_X,N_X)$ to the complex *-algebra of equivalence classes of bounded morphisms from $X$ to $\C$
	\item Send a morphism in $\strictems$ between two EMS $X$ and $Y$ to the morphism from $\linf(X)$ to $\linf(Y)$ which precomposes just like the above functor.  The fact that this is well-defined, and indeed is a morphism in the target category follows from the above lemma.
\end{itemize}
Now, we come to a (perhaps surprising) result that equality almost everywhere is "too strong" for our functor $\linf$.
\begin{proposition}
	$\linf$ is not faithful
\end{proposition}
\begin{proof}
	We begin by constructing an enhanced measurable space $X$, and an endomorphism $X\xrightarrow{f}X$, such that $f(x)\neq x$ for almost every $x\in X$, but for any measurable subset $E\subset X$, $f^*(E)\triangle E$ is negligible.  The proof of this is far outside the scope of this work, see for example \cite{fremlin_2011} for more.
\end{proof}

\subsection{Problem 4 - Alan Bohnert}

\subsubsection{Construct a functor $\textsf{Ban}^{op}$ to $\textsf{Ball}$}
Objects in $\textsf{Ban}^{op}$ are real (or complex) vector spaces with norms, and morphisms are $\R$ (or $\C$) linear maps:
\begin{equation*}
	(X,\|\cdot\|_X)\xrightarrow{f}(Y,\|\cdot\|_Y)
\end{equation*}
such that for all $x\in X$, $\|f(x)\|_Y\leq\|x\|_X$\\
Objects in $\textsf{Ball}$ are pairs $(V,B)$ where $V$ is a Hausdorff, locally convex topological real (complex) vector space, and $B$ a compact, convex, Hausdorff topological vector subspace of V, which is balanced.  Here, "balanced" means:
\begin{equation*}
	0\in B\wedge\forall x\in B \forall\;t\in\R (|t|\leq 1\implies tx\in B)
\end{equation*}
For a given object $(X,\|\cdot\|_X)$ in $\textsf{Ban}^{op}$, $X^*$ denotes the space of continuous linear functionals on $X$ with the weak-$*$ topology and $X^*_{\leq1}$ denotes the subspace of $X^*$ consisting of functionals of norm at most $1$. 
$(X^*,X^*_{\leq1})$ is an object of $\textsf{Ball}$.
Let $F: \banop\rightarrow\ball$ be the functor which sends $(X,\|\cdot\|_X)$ to $(X^*,X^*_\leq{1})$ and sends a morphism $g$ from $X\rightarrow{Y}$ to $ X^*\rightarrow{Y^*}$ to $F(g) = -\circ{g}$.
The functor $F$ is encoded in the following commutative diagram:

\begin{center}
	\begin{tikzcd}
		{(X,\|\cdot\|_X)} \arrow[dd, "g"']                &  & {(X^*,X^*_{\leq1})}                                        \\
		{} \arrow[rr, "F(g) = -\circ{g}"', maps to]       &  & {}                                                         \\
		{(Y,\|\cdot\|_Y)} \arrow[dd, "h"']                &  & {(Y^*,Y^*_{\leq1})} \arrow[uu, "F(g)(y^*) = y^*\circ{g}"'] \\
		{} \arrow[rr, "F(h) = -\circ{h}"', maps to]  &  &   {}                                                         \\
		{(Z,\|\cdot\|_Z)}  &  & {(Z^*,Z^*_{\leq{1}})} \arrow[uu, "F(h)(x^*)=x^*\circ{h}"']  
	\end{tikzcd}
\end{center}
From this diagram, we can verify
\begin{itemize}
	\item That $F(g)$ is actually a morphism in $\ball$
	\item $F$ is functorial
	\item $F$ is unital
\end{itemize}

\subsubsection{Construct a functor $\textsf{Ball}$ to $\textsf{Ban}^{op}$ }

Let $F:\ball \rightarrow \banop$ be the functor which sends an object $(V,A)$ to $(V',\|\cdot\|_A)$ where $V'$ is the space of continuous linear functionals on $V$ and $\|h\|_A=\sup_{x\in A}|h(x)|$ as required by the question.
Additionally, $F$ sends morphisms $g:(V,A)\rightarrow (W,B)$ to $F(g)= - \circ{g}$.
The functor $F$ is encoded in the following commutative diagram:

\begin{center}
	\begin{tikzcd}
		{(V,A)} \arrow[dd, "g"'] & & {(V',\|\cdot\|_A)} \\
		
		{} \arrow[rr, "F(g) = -\circ{g}"', maps to]& & {}\\
		
		{(W,B)} \arrow[dd, "h"'] & & {(W',\|\cdot\|_B)} \arrow[uu, "F(g)(k')= k' \circ g"']\\
		
		{} \arrow[rr, "F(h) = -\circ{g}"', maps to]{} & & {}\\
		
		{(X, C)} & & {(X',\|\cdot\|_C)} \arrow[uu, "F(h)(k')= k' \circ h"']\\
	\end{tikzcd}
\end{center}

\noindent From this diagram, we can verify
\begin{itemize}
	\item That $F(g)$ is a morphism in $\textsf{Ban}^{op}$
	\item $F$ is functorial
	\item $F$ is unital
\end{itemize}

\subsubsection{Prove monomorphisms in $\textsf{Ban}^{op}$ are precisely the injective maps}

\begin{proof}
	
	All injective maps are by definition monomorphisms.
	To prove the other direction, let $f:A\rightarrow B$ be a monomorphism in $\ban$, and $K$ be the kernel of $f$.
	Now, given the inclusion map $\iota:K\rightarrow A$ and the zero map $0:K\rightarrow A$ we have the following diagram.
	\begin{center}
		\begin{tikzcd}
			{K} \arrow[hookrightarrow, shift left=1ex, "\iota"]{r} \arrow[shift right=1ex, "0"']{r} & {A} \arrow[>->, "f"']{r} & {B}
		\end{tikzcd}
	\end{center}
	
	\noindent We note $f\circ 0(x)=f\circ \iota(x)$ for all $x\in K$.
	Since $f$ is a monomorphism, it follows that $0(x)=\iota(x)$ and the kernel of $f$ is trivial.
	This $f$ is injective.
	
	
\end{proof}

\subsubsection{Prove epimorphisms in $\textsf{Ball}$ are precisely the surjective maps}

\begin{proof}
	
	All surjective maps are by definition epimorphisms.
	To prove the other direction we suppose $f$ is an epimorphism in $\textsf{Ball}$.
	Let $h$ be the quotient map defined by $h:b \mapsto b+f(X)$ and $k$ be the quotient map defined by $k:b \mapsto 0+f(X)$. Then we have the following diagram.
	
	\begin{center}
		\begin{tikzcd}
			{(X,A)} \arrow["f"']{r} & {(Y,B)} \arrow[shift left=1ex, "h"]{r} \arrow[shift right=1ex, "k"']{r} & {(Y/f(X),B/f(X))}\\
		\end{tikzcd}
	\end{center}
	
	\noindent As $h$ and $k$ are quotient maps operating on Hausdorff Topological spaces, we know all properties are preserved such that $(Y/f(X),B/f(X))$ is a unit ball.
	Let $p,q\in X$ such that $h\circ f(p)=k\circ f(q)$.
	Since $f$ is an epimorphism $h=k$.
	So, for all $y\in Y$ we know $y+f(X)=0+F(X)$.
	Therefore $y$ is in the image $f(X)$.
	Thus every element in $Y$ has a preimage and $f$ is surjective on X.
	An identical argument will show the same result for $f$ being injective on $A$.
	
\end{proof}

\subsubsection{Extension of a linear functional given an inclusion of Banach spaces}

\begin{proof}
	
	Recalling the functor from $\textsf{Ban}^{op}$ to $\textsf{Ball}$ discussed in part 1, we apply the conditions of the question and have the resulting diagram.
	
	\begin{center}
		\begin{tikzcd}
			{(A,\|\cdot\|_X)} \arrow[hookrightarrow, "g"']{dd} &  & {(A^*,A^*_{\leq1})} \\
			
			{} \arrow[rr, "F(g) = -\circ{g}"', maps to] &  & {} \\
			
			{(B,\|\cdot\|_X)} &  & {(B^*,B^*_{\leq1})} \arrow[uu, "F(g)(b^*) = b^*\circ{g}"'] \\
			
		\end{tikzcd}
	\end{center}
	
	\noindent By equivalence the morphism $F(g)$ is an epimorphism in $\textsf{Ball}$.
	By part 4 this morphism is a surjective map.
	So, for every $a^* \in (A^*,A^*_{\leq1})$ there exists a nonempty preimage in $(B^*,B^*_{\leq1})$.
	Furthermore, $F(g)(b^*)=b*\circ{g}$ is a restriction of $b^*$ to $A$ since $g$ is the inclusion map and has norm $1$.
	Therefore, for all $a^*\in (A^*,A^*_{\leq1})$ there exists $b^*\in (B^*,B^*_{\leq1})$ such that is an extension of $a^*$ to $B$, and
	\begin{center}
		$\|b^*\|\leq\|a^*\|=\|b^*\circ{g}\|\leq \|b^*\|\|g\|=\|b^*\|\times 1=\|b^*\|$.
	\end{center}
	Consequently the extension $b^*$ has the same norm as $a^*$.
	
\end{proof}

\section{Problem Set 3}
\begin{enumerate}
	\item For which pairs of fields (of the same characteristic) does a categorical product exist?
	\item Prove that the category of connected topological spaces does not have coproducts.
	\item Prove: The category of Banach spaces with \textit{continuous maps} has no infinite coproducts.
	\item Prove:
	\begin{itemize}
		\item The category \textsf{TOSet} of totally ordered sets and order-preserving maps does not have coproducts.
		\item What about the category \textsf{WOSet} of well-ordered sets? 
	\end{itemize}
\item Prove: the category \textsf{TopGrp} of topological groups and continuous homomorphisms has coproducts.
\item Prove: the category of Lie groups (finite dimensional) does not have coproducts.
\item Investigate products and coproducts in the category \textsf{PG} of Hilbert spaces and contractive maps.
\item Express limits in analysis (i.e., in a given metric space) as categorical limits
\item Exhibit colimits of towers in $\textsf{Field}$
\item Prove or disprove: Given a tower in $\textsf{Man}$ whose morphisms are open embeddings, the colimit exists.  Give an example of a tower in $\textsf{Man}$ without a colimit.
\item Compute all natural transformations from $\textsf{Id}$ to $**$(double-dual) as endofunctors on $\textsf{Vect}_{\mathbb{k}}$ 	
\end{enumerate}
\subsection*{Assignments}
\begin{itemize}
	\item Problem 1 - Orin
	\item Problem 2 - Orin
	\item Problem 3 - Alan
	\item Problem 4 - JJ
	\item Problem 5 - Unassigned
	\item Problem 6 - Unassigned
	\item Problem 7 - Unassigned
	\item Problem 8 - Unassigned
	\item Problem 9 - Unassigned
\end{itemize}
\subsection{Problem 1 - Orin}
\begin{lemma}
	Any non-trivial field homomorphism is injective and preserves prime subfield up to isomorphism.  Any homomorphism between fields of different characteristic must be trivial.
\end{lemma}
\begin{proof}
	Let $f: K\rightarrow{S}$ be a field homomorphism which is not injective.  That is, there exists a nonzero $s\in S$ such that $f(s)=0$.  But then for any $b\in S$ we have: $$f(b) = f(b)f(1)=f(b)f(s\cdot{s^{-1}})=f(b)f(s)f(s^{-1})=0$$, so $f$ is trivial.  For the second statement, note that a nontrivial field homomorphism must send $1_{F_1}$ to $1_{F_2}$, and so must be an isomorphism of prime subfields.  Since field characteristic is an invariant on prime subfields, no nontrivial field homomorphisms exist between fields of different characteristic. 
\end{proof}
\begin{proposition}
	If $K$ and $L$ are fields such that their pairwise categorical product exists, then they have the same characteristic and both have trivial automorphism groups.
\end{proposition}
\begin{proof}
	The first part of the proof follows from the foregoing lemma.  For the second part, assume towards a contradiction that $K\times L$ exists and that one of the fields (denoted, WLOG, $K$) has a nontrivial automorphism.  Let $P$ denote the prime subfield of $K$ and $L$.  Then $P\xrightarrow{f_1}K$ and $P\xrightarrow{f_2}L$ give rise to another homomorphism $P\xrightarrow{f}K\times L$.  Consider the homomorphism $f': P\xrightarrow{f}K\times L$ given by $(\gamma\circ f_1)\times(f_2)$, where $\gamma$ is any non-trivial field automorphism on $K$.  By uniqueness, $f'\neq f$, but they must both yield $f_2$ when post-composed with $K\times L\xrightarrow{\pi_L}L$.  Thus, for some $x\in P$, $\pi_L\circ{(f-f')}(x)=0$ and $\pi_K\circ{(f-f')}(x)\neq 0$.  Then $(f-f')(x)\neq 0$, so there exists some $\frac{1}{(f-f')(x)}\in K\times L$, but $\pi_L (\frac{1}{(f-f')(x)}) = \frac{1}{0}$, contradiction.
\end{proof}
\subsection{Problem 2 - Orin}
\begin{proposition}
	The category of connected topological spaces with continuous maps has no coproduct
\end{proposition}
\begin{proof}
	The coproduct, if it existed, would have to be preserved under the forgetful functor to the usual category of topological spaces.  The coproducts in this case would be disjoint unions of sets.  But no disjoint union of sets is connected.  Take a family of topological spaces $X_\alpha$ over some index set $A$.  Then consider the coproduct: $X := \coprod(X_\alpha)_{\alpha\in A}$ together with the inclusion maps: $$\iota_\alpha: X_\alpha\to X$$.  For any $\alpha\in A$, it becomes clear that:
	$$(\cup_{\beta\neq\alpha}(\iota_\beta(X_\beta)))\cup \iota_\alpha(X_\alpha) = X$$ and
	$$(\cup_{\beta\neq\alpha}(\iota_\beta(X_\beta)))\cap \iota_\alpha(X_\alpha) = \varnothing$$.  Furthermore, a set in $X$ is open iff it's preimage under every $\iota_\alpha$ is open.  If $\omega$ is any given element in $A$, then
	$$\iota_\omega^*(\cup_{\alpha\in A}X_\alpha)=X_\omega$$. Since the $X_\alpha$ are all open in their respective topologies, we have in particular that $\cup_{\beta\neq\alpha}(\iota_\beta(X_\beta))$ is open, so we have a separation of $X$ and the proof is complete.
\end{proof}
\subsection{Problem 3 - Alan}

\begin{definition}
	Let $\textsf{C}$ be a concrete category with faithful functor $f:\textsf{C} \xrightarrow{}\textsf{Set}$ and $X\in\textsf{Ob(Set)}$.
	A \textit{Free Object} on $X$ is a pair $(F(X),i)$ of an object $F(X)\in\textsf{C}$ and an injective morphism $i:X\xrightarrow{}f(F(x))$ that satisfies the following universal property:
	
	\vspace{2mm}
	
	For any $B\in\textsf{C}$ and any morphism $\varphi : X\xrightarrow{}f(B)$, there exists a unique morphism $g:F(X)\xrightarrow{}B$ such that $\varphi=f(g)\circ i$.
	In other words, the following diagram commutes.
\end{definition}

\begin{center}
	\begin{tikzcd}
		{X} \arrow["i"]{rr} \arrow["\varphi"']{rd} & & {f(F(x))} \arrow["f(g)"]{ld}\\
		& {f(B)} &\\
	\end{tikzcd}
\end{center}

\begin{definition}
	Let $U:\textsf{C}\xrightarrow{}\textsf{Set}$ denote the forgetful functor on $\textsf{C}$.
\end{definition}

\begin{definition}
	The \textit{Free Functor} $F:\textsf{Set}\xrightarrow{}\textsf{C}$ is the left adjoint to the forgetful functor.
	Note, $F$ maps a set to its free object's component in $\textsf{C}$.
\end{definition}
\begin{lemma}\label{l}
	In a locally small concrete category where the faithful functor is the forgetful functor $U$, the free object of a coproduct of sets is the coproduct of the free objects of those sets. 
\end{lemma}
\begin{proof}
	Let $C$ be a concrete category with faithful functor being the forgetful $U:\textsf{C}\xrightarrow{}\textsf{Set}$.
	Let $X_i\in Obj(\textsf{Set})$ for all $i\in I$, $C\in \textsf{C}$, and $F:\textsf{Set}\xrightarrow{}\textsf{C}$ be the free functor.
	By the definition of coproducts we know $$\hom(\coprod_{i\in I} F(X_i), C) 
	\cong \prod_{i\in I} \hom(F(X_i),C).$$
	Since the free functor and the forgetful functor are an adjunction, we then have $$\prod_{i\in I} \hom(F(X_i),C) \cong \prod_{i\in I} \hom(X_i,U(C)).$$
	By the definition of products $$\prod_{i\in I} \hom(X_i,U(C)) \cong \hom(\coprod_{i\in I} X_i, U(C)).$$
	Finally, the definition of free objects then tells us $$\hom(\coprod_{i\in I} X_i, U(C)) \cong \hom(F(\coprod_{i\in I}X_i),C).$$
	
	With this chain of isomorphisms we see $\coprod_{i\in I} F(X_i)$ and $F(\coprod_{i\in I}X_i)$ both objects represent the same functor $\textsf{C}^{\textsf{op}} \xrightarrow{}\textsf{Set}$.
	By the Yoneda Lemma $\coprod_{i\in I} F(X_i) \cong F(\coprod_{i\in I}X_i)$.
\end{proof}

\begin{proposition}
	The category of Banach spaces with \textit{continuous maps} has no infinite coproducts.
\end{proposition}

\begin{proof}
	Let $R$ denote the Banach space $(\R,|\cdot|)$ with absolute value as its norm.
	Seeking a contradiction, suppose infinite coproducts exist in $\textsf{Ban}$ exist.
	Then $\coprod_{i\in \N} R_i$, the coproduct of $\N$ copies of R, exists.
	
	\vspace{3mm}
	
	By the definition of free objects, $R$ is the free $\textsf{Ban}$ object of a singleton set $\{\ast\}$.
	Consequently, lemma \ref{l} tells us $\coprod_{i\in \N} R_i \cong F(\coprod_{i\in \N} \{\ast\}_i)$.
	In $\textsf{Set}$ the coproduct is the disjoint union, and when all the sets are singleton sets the coproduct is simply the indexing set.
	Given the indexing set is $\N$, we have $\coprod_{i\in \N} R_i \cong F(\N)$.
	
	\vspace{3mm}
	
	It then follows the free object $(F(\N),i)$ exists.
	Let $\|\cdot\|$ denote the norm on $F(\N)$, and define $\varphi(n):=n\|i(n)\|$ for all $n\in\N$.
	Since $F(\N)$ is a free object, there exists a unique $g:F(\N)\xrightarrow{} R$ such that this diagram commutes.
	
	\begin{center}
		\begin{tikzcd}
			{\N} \arrow["i"]{rr} \arrow["\varphi"']{rd} & & {U(F(\N))} \arrow["U(g)"]{ld}\\
			& {\R} &\\
		\end{tikzcd}
	\end{center}
	
	As a morphim in $\textsf{Ban}$, we know $g$ is continuous and so bounded.
	So, there exists a constant $c\in \R$ such that $$|g(x)|\le c\|x\|$$ for all $x\in F(\N)$.
	We note, the image of $\N$ under $i$ is a subset of the vectors in $F(\N)$.
	Consequently the following must hold for all $n\in \N$.
	\begin{center}
		$|g(i(n)|\le c\|i(n)\|$\\
		\vspace{1mm}
		$\varphi(n) \le c\|i(n)\|$\\
		\vspace{1mm}
		$n\|i(n)\|\le c\|i(n)\|$\\
		\vspace{1mm}
		$n\le c$\\
	\end{center}
	This is a contradiction since $c\in \R$ is fixed and  the inequality must hold for all $n\in \N$.
	Therefore free objects do not exist in $\textsf{Ban}$ and consequently infinite coproducts do not exist in $\textsf{Ban}$ either.
\end{proof}

\subsection{Problem 6 - Bradley}

\begin{proposition}
$\mathbf{FinGrp}$ is a full subcategory of $\mathbf{FinLGrp}$, the category of finite Lie Groups.
\end{proposition}
\begin{proof}
We first note that for any finite group $A$, if we treat $A$ as a $0$-dimensional manifold, then we have that $A$ is automatically a Lie group; therefore, $\mathbf{FinGrp}$ is a subcategory of $\mathbf{FinLGrp}$. Furthermore, pick any two objects $A$,$B$ in $\mathbf{FinGrp}$ since morphisms in $\mathbf{FinLGrp}$ are smooth group homomorphisms, we have that any morphism $f:A\to B$ in $\mathbf{FinLGrp}$ is automatically in $\mathbf{FinGrp}$. Therefore, the inclusion functor $\iota : \mathbf{FinGrp}\to \mathbf{FinLGrp}$ is full and hence $\mathbf{FinGrp}$ is a full subcategory of $\mathbf{FinLGrp}$.
\end{proof}
Therefore, if we can manage to show that $\mathbf{FinGrp}$ does not have coproducts then we are done. Fortunately, it is true that $\mathbf{FinGrp}$ does not have coproducts; however, it's not enough to say it so let us prove it.

\begin{proposition}
$\mathbf{FinGrp}$ does not have coproducts.
\end{proposition}
\begin{proof}
Let us suppose to the contrary that $\mathbf{FinGrp}$ has coproducts. Consider the groups $\Z_{2}$ and $\Z_{2}$ along with inclusion maps $\iota_{1},\iota_{2}$ into the coproduct, $C$, of $\Z_{2}$ and $\Z_{2}$ and Let K be any other finite group that is generated by two elements $\alpha, \beta$ both of which have order 2. Then consider the following diagram, noting that both $f_{1}$ and $f_{2}$ are group homomorphisms because $f_{1}$ could send $0\mapsto 0$ and $1\mapsto \alpha$, similarly for $f_{2}$.
\begin{center}
    % https://tikzcd.yichuanshen.de/#N4Igdg9gJgpgziAXAbVABwnAlgFyxMJZARgBoAmAXVJADcBDAGwFcYkQAdDgW3pwAsARoOAAtAL4B9YOXEhxpdJlz5CKMsWp0mrdgGF5ikBmx4CRcqU00GLNok48+QkROmzDS06qIAGUr5atroOovJaMFAA5vBEoABmAE4Q3Ej+IDgQSGQgjPSCMIwACspmaiCJWFH8OCA2OvaO+Dj00sRyCgnJqYiWGVmIOXkFxaU+Doww8bX1duxcza0yHUZJKWk0mUgAzDQFYFA76cGN8W0rXeu9mwO7IPuHiAC028cN7GfLdbn5hSXe5gclWqtU6IDWPRyW0Qd2GfzGgIqVRq3xOH3C4iAA
\begin{tikzcd}
K &                                                                        &                                                                         \\
  & C \arrow[lu, "f"']                                                     & \mathbb{Z}_{2} \arrow[l, "\iota_{2}"] \arrow[llu, "f_{2}"', bend right] \\
  & \mathbb{Z}_{2} \arrow[u, "\iota_{1}"'] \arrow[luu, "f_{1}", bend left] &                                                                        
\end{tikzcd}
\end{center}

Let us notice that the morphisms $\iota_{1}$ and $\iota_{2}$ must be nontrivial. To see this, suppose to the contrary that $\iota_{1}$ is the trivial morphism. Then, let $K$ be any group such that the map $f_{1}$ is nontrivial e.g. take $K=\Z_{2}$ then $f_{1}=\text{id}_{\Z_{2}}$. Then, there is no morphism $f$ such that $f_{1}=f\circ \iota_{1}$ since $f\circ \iota_{1}$ is the trivial map whereas $f_{1}$ is the identity. Therefore, the universal mapping property would not be satisfied, a contradiction. Similarly, $\iota_{2}$ is nontrivial as well.
Now, let $c_{1}=\iota_{1}(1)$ and $c_{2}=\iota_{2}(1)$ be elements of $C$ of order two. From the universal mapping property, we see that if $K$ is a finite group with elements $k_{1}$ and $k_{2}$ of order two, then there is a unique group homomorphism $f:C\to K$ such that $k_{1}=f(c_{1})=f(\iota_{1}(1))$ and $k_{2}=f(c_{2})=f(\iota_{2}(1))$. If $K$ is generated by the elements $k_{1}$ and $k_{2}$, then $f(C)=K$; therefore, $|C| \geq |K|$. Let us now recall that the dihedral group $\text{D}_{n}=\langle s,r \rangle$  is  generated by rotations, $r$, and reflections, $s$. Moreover, the dihedral group is also generated by the elements $s$ and $sr$ both of which are reflections and thus have order $2$. Therefore, it follows that if $K=\text{D}_{n}$ then $|C| \geq | \text{D}_{n}|$. Hence, $|C| \geq |2n|$ for all $n\in \N$. This; however, is a contradiction because $C$ must be a finite group. Therefore, $\mathbf{FinGrp}$ does not have coproducts and hence $\mathbf{FinLGrp}$ does not have coproducts as well.

\end{proof}

\section{Problem Set 4}
\subsection{Assignments}
\alert{Fix enumeration}
\begin{enumerate}
	\item \begin{itemize}[a.]
		\item Show that if $M$ is a f.g. projective $R$-module for some comm. ring $R$, show that $$\textsf{Hom}(M,-)\dashv M\otimes_R -$$
		\item Show that if $M$ is a (not necessarily f.g. and projective) $R$-module, then $\textsf{Hom}(M,-)$ does not preserve colimits in general
	\end{itemize}
	\item Recall that there exists a left adjoint functor $L$ to the forgetful functor from Locally Compact Topological Spaces to Set. We have seen that $L$ induces the discrete topology. Recall further that the $\pi_0$ functor is the left adjoint to $L$. Show that $\pi_0$ does not admit a left adjoint itself.
	\item Recall the functor Const: $C\to Fun(I,C)$. Show that $\text{colim}(Const X) = \bigsqcup_{\pi_0 I}X$.
\end{enumerate}
\bibliography{ctreferences}
\bibliographystyle{plain}
\end{document}
