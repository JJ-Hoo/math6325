\documentclass{article}

%packages
\usepackage[utf8]{inputenc}
\usepackage{amsmath}
\usepackage{amssymb}
\usepackage{amsfonts}
\usepackage{float}
\usepackage{amsthm}
\usepackage{outlines}
\usepackage{graphicx}
\usepackage{fullpage}
\usepackage{color}
\usepackage{hyperref}
\usepackage{mathtools}
\usepackage{easyReview}
\hypersetup{
	colorlinks=true,
	linkcolor=blue,
	urlcolor=red,
	linktoc=all
}
\definecolor{dg}{rgb}{0.0, 0.5, 0.0}

%Commands
\newcommand{\R}{\mathbb{R}}
\newcommand{\C}{\mathbb{C}}
\newcommand{\N}{\mathbb{N}}
\newcommand{\Q}{\mathbb{Q}}
\newcommand{\Z}{\mathbb{Z}}
\newcommand{\D}{\mathbb{D}}
\newcommand{\HH}{\mathbb{H}}

%Theorems, etc. (shared counters)
\newtheorem{theorem}[subsection]{Theorem}
\newtheorem{example}[subsection]{Example}
\newtheorem{proposition}[subsection]{Proposition}
\newtheorem{lemma}[subsection]{Lemma}
\newtheorem{definition}[subsection]{Definition}

%Math Operators
\DeclareMathOperator{\calg}{\textsf{CAlg}^{*}_{\mathbb{C}}}
\DeclareMathOperator{\linf}{L^\infty}

\title{Category Theory Problem Sets and Solutions}
\author{Fall 2022}
\begin{document}
	
	\maketitle
	\tableofcontents
	\newpage
	\section{Problem Set 1}
	\subsection*{Assignments}
	\begin{itemize}
		\item Problem 1 - Unclaimed
		\item Problem 2 - Emilio Verdooren
		\item Problem 3 - Emilio Verdooren
		\item Problem 4 - Orin Gotchey
		\item Problem 5 - Unclaimed
		\item Problem 6 - Alan Bohnert
		\item Problem 7 - James
		\item Problem 8 - James
		\item Problem 9 - Unclaimed
		\item Problem 10 - Unclaimed
		\item Problem 11 - Unclaimed
	\end{itemize}
	
	\subsection{Problem 4 - Orin Gotchey}
	\subsubsection{Measurable Spaces as a Category}
	\begin{definition}{$\sigma{}$-algebras}
		Let $X$ be a set.  Let $\Omega$ be any subset of $\mathcal{P}(X)$ satisfying the following conditions:
		\begin{itemize}
			\item $X\in\Omega_{X}$
			\item For each $E\in\Omega$, $X\backslash E\in\Omega_{X}$
			\item For any index $I: \mathbb{N}\rightarrow\Omega_{X}$, $(\cup_{n\in\mathbb{N}}I(n))\in\Omega_{X}$
		\end{itemize} $\Omega$ is called a $\sigma$\textit{-algebra} on $X$, the pair $(X,\Omega_{X})$ a \textit{measurable space}, the elements of $\Omega_{X}$ the \textit{measurable subsets} of $X$.  
	\end{definition}
	It follows immediately that $\varnothing\in\Omega_{X}$, and that $\Omega_{X}$ is closed under countable intersection.\\
	\begin{definition}{Measurable Maps}The maps $f: X\rightarrow Y$ between measurable spaces which have the following property: 
		\begin{equation*}
			\forall E\subset X: (f(E)\in\Sigma\implies E\in\Omega) 
		\end{equation*}
		are called \textit{measurable maps} or \textit{measurable functions}.
	\end{definition}
	Let $\texttt{Meas}$ be the category specified as follows:
	\begin{outline}
		\1 Objects are the measurable spaces $(X,\Omega)$
		\1 Morphisms are the measurable functions.
	\end{outline}
	Then, given any morphism $f: X\rightarrow{}Y$,
	\begin{equation*}
		f\circ\texttt{Id}_X = f
	\end{equation*}
	\begin{equation*}
		\texttt{Id}_Y\circ{}f = f
	\end{equation*}
	Associativity follows from the fact that the composition of functions on the underlying sets is associative.\\Given two composable morphisms, say $f: X\rightarrow{}Y$ and $g:Y\rightarrow{}Z$, consider the composition $g\circ f: X\rightarrow{}Z$, and let $\gamma\in\Omega_{Z}$.  Then:
	\begin{equation*}
		g^{-1}(\gamma) \in \Omega_{Y}
	\end{equation*}
	\begin{equation*}
		f^{-1}(g^{-1}(\gamma)) = (g\circ f)^{-1}(\gamma) \in\Omega_{X}
	\end{equation*}.
	Thus, we have that \texttt{Meas} is a category.
	
	\subsubsection{Enhanced Measurable Spaces}
	Let $(X,\Omega_X)$ be a topological space.  Then $\mathcal{P}(X)$ forms a Boolean commutative ring with the operations $\cap$ and $\triangle$ as multiplication and addition, respectively, and of which $\Omega_X$ is a subring.
	Define an \textit{enhanced measurable space} as a triple $(X,\Omega_X,N_X)$, where $(X,\Omega_X)$ form a measurable space, and $N_X$ is a $\sigma$-ideal of $\Omega_{X}$ (recall: a $\sigma$-\textit{ideal} is an ideal which is closed under \textit{countable} addition).  A \textit{negligible set} in $X$ is some subset of $N_{X}$.\\The \textit{measurable maps} $f: (X,\Omega_X, N_X)\rightarrow(Y,\Omega_Y,N_Y)$ are maps of sets: $f: X_f\rightarrow{}Y$, where $X_f\subset{}X$, such that $f$ obeys the following conditions (which are verified for the identity maps in the subpoints where $X=Y$):
	\begin{enumerate}
		\item The set $X\backslash{}X_f$ is negligible
		\begin{itemize}\item $X=X_{\texttt{Id}_X}$ and $X\backslash{}X_{\texttt{Id}_X}=\varnothing\in{}N_X$, by definition of ideal. \end{itemize}
		\item For any $m_{y}\in\Omega_Y$, there exists a set $m_{x}$ such that $f^{-1}(m_y)\triangle{}m_x$ is negligible
		\begin{itemize}\item Given $m_{x}$, $\texttt{Id}^{-1}_X(m_x)\triangle{}m_X = m_x\triangle{}m_x = \varnothing\in{}N_X$ \end{itemize}
		\item For any $n_y\in N_Y$, the set $f^{-1}(n_y)$ is negligible.
		\begin{itemize} \item $\texttt{Id}_X(n_x) = n_x$ \end{itemize}
	\end{enumerate}
	We cannot define composition of morphisms strictly as composition of underlying maps, because there is no guarantee, e.g., for two maps between enhanced measurable spaces $f:X\rightarrow{}Y,\,g:Y\rightarrow{}Z$, that $\textit{Im}f\subset{}Y_g$.  Thus, we restrict the domain of the composition to:
	\begin{equation*}
		X_{g\circ{}f} := f^{-1}(Y_g)
	\end{equation*}.
	However, it is clear by inspection that composition of morphisms retains associativity.
	Then,
	\begin{equation*}
		X\backslash{}X_{g\circ{}f} = X\backslash{}f^{-1}(Y_g) = (f^{-1}(Y\backslash(Y_g)))
	\end{equation*}
	The negligibility of the above quantity then follows from the definition of $f$.\\
	Furthermore, given $m_z\in\Omega_Z$, we have that $(g\circ{}f)^{-1}(m_z) = f^{-1}(g^{-1}(m_z))$.  Since $g$ is measurable (\textit{why?}) and since $f$ is presumed to satisfy (2), $g\circ{}f$ satisfies (2).\\(3) is clearly transitive.\\Thus, enhanced measurable spaces and measurable maps form a category.
	\subsubsection{Equality Almost Everywhere}
	Two parallel morphisms $f,g: (X,\Omega_X, N_X)\rightarrow{}(Y,\Omega_Y,N_Y)$ are "equal almost everywhere" if the set $\{x\in X_f\cap{}X_G : f(x)\neq{}g(x)\}$ is negligible.  Let "$f$ and $g$ are equal almost everywhere" be denoted $f\sim{}g$.  Claim: $\sim$ defines an equivalence relation.
	\begin{itemize}
		\item Reflexivity: A function differs from itself on the empty set ($\varnothing$), which is negligible (see above)
		\item Symmetry: Note that the symbols $f$ and $g$ in the definition of equality almost everywhere are symmetric
		\item Transitivity: If $f\sim{}g$ and $g\sim{}h$ for parallel morphisms $f,g,$ and $h$, then 
		\begin{equation*}
			\{x\in X_f\cap{}X_h:f(x)\neq h(x)\}\subset{}(\{x\in X_f\cap{}X_g: f(x)\neq g(x)\}\cup{}\{x\in{}X_g\cap{}X_h: g(x)\neq{}h(x)\})
		\end{equation*}, and $N$ is closed under countable unions and taking subsets, so the left hand side of the above is negligible.
	\end{itemize}
	Furthermore, this equivalence relation is compatible with composition.  Assume that there are morphisms $f,f': X\rightarrow{}Y$ and $g,g': Y\rightarrow{}Z$. such that $f\sim{}f'$ and $g\sim{}g'$.  We're interested in the set
	\begin{multline*}
		\{ x\in{}X_{g\circ{}f}\cap X_{g'\circ{}f'} : (g\circ{}f)(x)\neq(g'\circ{}f')(x) \} \subset{} \{x\in{}X_f\cap{}X_{f'} : f(x)\neq{}f'(x)\}\\ \cup{}f^{-1}(\{y\in{}Y_g\cap{}Y_g': g(y)\neq{}g'(y)\})
	\end{multline*}
	This set is the union of two negligible sets.
	\subsubsection{Hom-sets mod an Equivalence Relation}
	Suppose that for every pair of objects $X,Y$ in a category $C$, we are given (e.g. by the above) an equivalence relation $R_{X,Y}$ on $C(X,Y)$ that is compatible with composition (i.e. if $f\sim_Rf'$ and $g\sim_Rg'$ then $(g\circ{}f)\sim_R(g'\circ{}f')$.  We identify all morphisms in $C$ between any two objects $X$ and $Y$ which relate through $R_{X,Y}$.  Composition of equivalence classes of $\sim{}$ does not depend on choice of representative: this is exactly compatibility with $\circ{}$\\Verifying that the proper morphisms are unital and associative are gifted as simple exercises to the reader ;)
	
	
	
	\subsection{Problem 6 - Alan Bohnert}
	\subsubsection{Question}
	Fix a category \textbf{C}. 
	A \textit{bimorphism} in \textbf{C} is a morphism $f$ that is simultaneously a monomorphsim and an epimorphism. Is any isomorphism a bimorphism? 
	Give and example of a category \textbf{C} and a bimorphism $f$ in \textbf{C} that is not an isomorphism.
	
	\subsubsection{Solution}
	In any category \textbf{C} every isomorphism is a bimorphism.
	
	\begin{proof}
		Let $f:X\xrightarrow{}Y$ be an isomorphism in \textbf{C}. 
		Then there exists a morphism $g:Y\xrightarrow{}X$ in \textbf{C} such that 
		\begin{center}
			$gf=$Id$_X$ and $fg=$Id$_Y$.
		\end{center}
		
		To show $f$ is a monomorphism let $h,k:W \rightrightarrows X$ and $fh=fk$.
		It follows that $gfh=gfk$ for the $g$ given above.
		Therefore Id$_X h=$Id$_X k$ and so $h=k$ tells us $f$ is a monomorphism.
		
		To show $f$ is an epimorphism let $m,n:Y \rightrightarrows Z$ and $mf=nf$.
		Composing with the $g$ we know $mfg=nfg$.
		Consequently $m$Id$_Y=n$Id$_Y$ and $m=n$ tells us $f$ is an epimorphism.
		Therefore $f$ is a bimorphism.
	\end{proof}
	
	\vspace{2mm}
	
	\noindent Let \textbf{C} be the category \textbf{Ring} and let $f:\Z \xhookrightarrow[]{} \Q$ the inclusion map.
	We claim $f$ is a bimorphism but not an isomorphism.
	
	\begin{proof}
		To show $f$ is a monomorphism let $h,k:W \rightrightarrows \Z$ and $f\circ h(w)=f \circ k (w)$ $\forall w\in W$.
		Since $f$ is injective
		\begin{center}
			$h(w)=f\circ h(w)=f\circ k(w)=k(w)$.
		\end{center}
		Therefore $h(w)=k(w)$ and $f$ is a monomorphism.
		
		To show $f$ is an epimorphism let $m,n:\Q \rightrightarrows S$ such that $m\circ f(x)=n\circ f(x)$ $\forall x\in \Q$.
		Since $f$ is injective, we know $m(z)=n(z)$ $\forall z\in\Z$.
		Seeking a contraction, suppose there exists $\frac{a}{b}\in \Q$ such that $m(\frac{a}{b})\neq n(\frac{a}{b})$.
		Given $m$ and $n$ are ring homomorphisms we know
		\begin{center}
			$m(a)m(b^{-1})=m(\frac{a}{b}) \neq n(\frac{a}{b})=n(a)n(b^{-1})$.
		\end{center}
		Given $b$ is an invertible integer and $m(b)=n(b)$ we can multiply on the right and retain the inequality.
		Thus, 
		\begin{center}
			$m(a)m(b^{-1})m(b)\neq n(a)n(b^{-1})n(b)$
		\end{center} 
		and as ring homomorphisms we have
		\begin{center}
			$m(a)=m(a)m(b^{-1}b)\neq n(a)n(b^{-1}b)=n(a)$.
		\end{center}
		Therefore $f$ is a epimorphism.
		
		To show $f$ is not an isomorphism we note $\frac{1}{3}\in \Q$ has no preimage in $\Z$.
	\end{proof}
\section{Problem Set 2}
\subsection*{Assignments}
\begin{itemize}
	\item Problem 1 - Orin Gotchey
	\item Problem 2 - James
	\item Problem 3 - Bradley
	\item Problem 4 - Alan
	\item Problem 5 - James
	\item Problem 6 - Emilio
\end{itemize}
\subsection{Problem 1 - Orin Gotchey}
\begin{lemma}{Existence of Borel Algebras.}
	Let $X$ be a topological space.  Then there exists a unique $\sigma$-algebra, $\Omega$ on $X$ which contains all open subsets of $X$ and which is smallest with respect to inclusion. 
\end{lemma}
\begin{proof}
	Let $\Sigma$ be the collection of all $\sigma$-algebras on $X$ which contain all open subsets of $X$.  $\Sigma$ contains $\mathcal{P}(X)$, and thus is nonempty.	Let
	\begin{equation*}
	\Omega := \cap_{x\in\Sigma}x
	\end{equation*}
	Clearly, $X\in\Omega$. Given an index $I: \mathbb{N}\rightarrow\Omega$, such that for every natural $n$, $I(n)\in\Omega$, we have that $I(n)\in{x},\;\forall{x}\in\Sigma$, whence it follows that $\cap_{n\in\mathbb{N}}I(n)\in{x},\;\forall{x}\in\Sigma$.  Therefore, $\cap_{n\in\mathbb{N}}I(n)\in{\Omega}$.  By a similar argument, for any $E\in\Omega$, $X\backslash{E}\in\Omega$.  Thus, $\Omega$ contains all open subsets of $X$, and is indeed inferior to any other $\sigma$-algebra with this property.  
\end{proof}
\begin{definition}
	A \textit{complex *-algebra} $A$ is a complex algebra, equipped with a complex-antilinear operation $*:A\rightarrow{A}$ obeying the following:
	\begin{equation*}
		(ab)^* = b^*a^*
	\end{equation*}
	\begin{equation*}
		1^* = 1
	\end{equation*}
	\begin{equation*}
		(a^*)^* = a
	\end{equation*}
\end{definition}
\begin{definition}
	A morphism $f: \calg\rightarrow{}\C$ is called "bounded" if it factors through some bounded subset of $\C$ \add{explanation of factoring} 
\end{definition}
\begin{proposition}
	Together with morphisms: $f: A\rightarrow{B}$ satisfying $f(a^*) = f(a)^*$, and objects: commutative complex *-algebras, $\calg$ is a category.
\end{proposition}
	Let $\textsf{L}^\infty: \textsf{PreEMS}^{op}\rightarrow{\calg}$ send an enhanced measurable space to the complex *-algebra of bounded morphisms: $(X,\Omega_X,N_X) \mapsto (\linf(X): \{\phi: (X,\Omega_X,N_X)\rightarrow(\C,\Omega_\C, \{\varnothing\})|\,\phi\,\text{bounded}\})$, and which sends an enhanced measurable morphism $f: (X,\Omega_X,N_X)\rightarrow{}(Y,\Omega_Y,N_Y)$ to \begin{equation*}
\textsf{L}^\infty(f) : (\linf(Y) : \{\phi:(Y,\Omega_Y,N_Y)\rightarrow(\C,\Omega_{\C},N_\C)\})\rightarrow{}(\linf(X) : \{\psi : (X,\Omega_X,N_X)\rightarrow(\C,\Omega_\C,N_\C)\})
	\end{equation*}
given by:
\begin{equation*}
	(\linf(f))(\phi) = (\phi\circ{}f)
\end{equation*}
To show:
\begin{itemize}
	\item $\linf(f)$ defines a morphism in $\calg$
	\item $\linf$ respects identity
	\item $\linf$ respects composition
\end{itemize}
\end{document}
